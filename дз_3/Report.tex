\documentclass[a5paper, 10pt]{article}

% Текст
\usepackage[utf8]{inputenc} % UTF-8 кодировка
\usepackage[russian]{babel} % Русский язык
\usepackage{indentfirst} % красная строка в первом параграфе в главе
% Отображение страниц
\usepackage{geometry} % размеры листа и отступов
\geometry{
	left=12mm,
	top=25mm,
	right=15mm,
	bottom=17mm,
	marginparsep=0mm,
	marginparwidth=0mm,
	headheight=10mm,
	headsep=7mm,
	nofoot}
\usepackage{afterpage,fancyhdr} % настройка колонтитулов
\pagestyle{fancy}
\fancypagestyle{style}{ % создание нового стиля style
	\fancyhf{} % очистка колонтитулов
	\fancyhead[LO, RE]{Домашняя работа №3} % название документа наверху
	%\fancyhead[RO, LE]{\leftmark} % название section наверху
	\fancyfoot[RO, LE]{\thepage} % номер страницы справа внизу на нечетных и слева внизу на четных
	\renewcommand{\headrulewidth}{0.25pt} % толщина линии сверху
	\renewcommand{\footrulewidth}{0pt} % толцина линии снизу
}
\fancypagestyle{plain}{ % создание нового стиля plain -- полностью пустого
	\fancyhf{}
	\renewcommand{\headrulewidth}{0pt}
}
\fancypagestyle{title}{ % создание нового стиля title -- для титульной страницы
	\fancyhf{}
	\fancyhead[C]{{\footnotesize
			Министерство образования и науки Российской Федерации\\
			Федеральное государственное автономное образовательное учреждение высшего образования
	}}
	\fancyfoot[C]{{\large 
			Санкт-Петербург, 2023-2024
	}}
	\renewcommand{\headrulewidth}{0pt}
}

% Математика
\usepackage{epigraph}
\usepackage{cancel}
\usepackage{amsmath, amsfonts, amssymb, amsthm} % Набор пакетов для математических текстов
%\usepackage{dmvnbase} % мехматовский пакет latex-сокращений
\usepackage{cancel} % зачеркивание для сокращений
% Рисунки и фигуры
\usepackage[pdftex]{graphicx} % вставка рисунков
\usepackage{wrapfig, subcaption} % вставка фигур, обтекая текст
\usepackage{caption} % для настройки подписей
\captionsetup{figurewithin=none,labelsep=period, font={small,it}} % настройка подписей к рисункам
% Рисование
\usepackage{tikz} % рисование
\usepackage{circuitikz}
\usepackage{pgfplots} % графики
% Таблицы
\usepackage{multirow} % объединение строк
\usepackage{multicol} % объединение столбцов
% Остальное
\usepackage[unicode, pdftex]{hyperref} % гиперссылки
\usepackage{enumitem} % нормальное оформление списков
\setlist{itemsep=0.15cm,topsep=0.15cm,parsep=1pt} % настройки списков
% Теоремы, леммы, определения...
\theoremstyle{definition}
\newtheorem{Def}{Определение}
\newtheorem*{Axiom}{Аксиома}
\theoremstyle{plain}
\newtheorem{Th}{Теорема}
\newtheorem{Lem}{Лемма}
\newtheorem{Cor}{Следствие}
\newtheorem{Ex}{Пример}
\theoremstyle{remark}
\newtheorem*{Note}{Замечание}
\newtheorem*{Solution}{Решение}
\newtheorem*{Proof}{Доказательство}
% Свои команды
\newcommand{\comb}[1]{\left[\hspace{-4pt}\begin{array}{l}#1\end{array}\right.\hspace{-5pt} } % совокупность уравнений
% Титульный лист
\usepackage{csvsimple-l3}
\newcommand*{\titlePage}{
	\thispagestyle{title}
	\begingroup
	\begin{center}
		%		{\footnotesize
			%			Министерство образования и науки Российской Федерации\\
			%			Федеральное государственное автономное образовательное учреждение высшего образования
			%		}
		%		
		\vspace*{6ex}
		
		{\small
			САНКТ-ПЕТЕРБУРГСКИЙ НАЦИОНАЛЬНЫЙ ИССЛЕДОВАТЕЛЬСКИЙ УНИВЕРСИТЕТ ИТМО	
		}
		
		\vspace*{2ex}
		
		{\normalsize
			Факультет систем управления и робототехники
		}
		
		\vspace*{15ex}
		
		{\Large \bfseries 
			Домашняя работа №3
		}
\vspace*{3ex}
		
		{ \Large 
			Интегрирующий множитель и замены в ДУ 1-го порядка
		}
\vspace*{3ex}
		
		{  \bfseries 
			по дисциплине Дифференциальные уравнения
		}
	\end{center}
	\vspace*{20ex}
	\begin{flushright}
		{\large 
			\underline{Выполнила}: студентка гр. \textbf{R3238}\\
			\begin{flushright}
				\textbf{Нечаева А. А.}\\
			\end{flushright}
		}
		
		\vspace*{5ex}
		
		{\large 
			\underline{Преподаватель}: \textit{Бойцев Антон Александрович}
		}
	\end{flushright}	
	\newpage
	\setcounter{page}{1}
	\endgroup}

\begin{document}
	\titlePage
	\pagestyle{style}
	
\newpage
\epigraph{Трудность решения в какой-то мере входит в само понятие задачи: там, где нет трудности, нет и задачи.}{Д. Пойа}

\section{Решить уравнение, подобрав интегрирующий множитель}
\begin{equation*}
\left( 2xy + x^2y+\frac{y^3}{3} \right)dx + \left(x^2 + y^2 \right)dy = 0
\end{equation*}
Записать общий интеграл уравнения в виде $F(x, y) = C$.\\\\
\textit{\textbf{Решение:}}\\
Запишем уравнение для нахождения интегрирующего множителя
\begin{equation*}
\mu'_y \left( 2xy + x^2y+\frac{y^3}{3} \right) + \mu \left( 2xy + x^2y+\frac{y^3}{3} \right)'_y = \mu'_x \left(x^2 + y^2 \right) + \mu \left(x^2 + y^2 \right)'_x
\end{equation*}
\begin{equation*}
\mu'_y \left( 2xy + x^2y+\frac{y^3}{3} \right) + \mu \left(\bcancel{ 2x} + x^2+ y^2 \right) = \mu'_x \left(x^2 + y^2 \right) + \bcancel{2x \mu} 
\end{equation*}
\begin{equation*}
\mu'_y \left( 2xy + x^2y+\frac{y^3}{3} \right) + \mu \left(x^2+ y^2 \right) = \mu'_x \left(x^2 + y^2 \right)
\end{equation*}
Будем искать частное решение в виде $\mu = \mu (x)$, тогда $\mu'_y = 0$.
\begin{equation*}
 \mu \left(x^2+ y^2 \right) = \mu'_x \left(x^2 + y^2 \right) \left| : (x^2 + y^2) \neq 0 \right.
\end{equation*}
\begin{equation*}
 \frac{d\mu}{dx} = \mu
\end{equation*}
\begin{equation*}
\int \frac{d\mu}{\mu} = \int dx
\end{equation*}
\begin{equation*}
\ln | \mu | = x + C
\end{equation*}
Интегрирующий множитель:
\begin{equation*}
 \mu = e^x
\end{equation*}
Домножим исходное уравнение на найденный интегрирующий множитель, получим:
\begin{equation*}
e^x \left( 2xy + x^2y+\frac{y^3}{3} \right)dx + e^x \left(x^2 + y^2 \right)dy = 0
\end{equation*}
Покажем, что полученное уравнение является уравнением в полных дифференциалах:
\begin{equation*}
P(x, y) = e^x \left( 2xy + x^2y+\frac{y^3}{3} \right)
\end{equation*}
\begin{equation*}
Q(x, y) =  e^x \left(x^2 + y^2 \right)
\end{equation*}
\begin{equation*}
P'_y= e^x \left( 2x + x^2+y^2 \right)
\end{equation*}
\begin{equation*}
Q'_x =  e^x \left(x^2 + y^2 \right) + e^x \left(2x \right)  = e^x \left(2x + x^2 + y^2 \right) 
\end{equation*}
\begin{equation*}
P'_y= Q'_x 
\end{equation*}
Значит, по признаку полученное уравнение является уравнением в полных дифференциалах и имеет вид:
\begin{equation*}
\frac{\partial F}{\partial x} dx + \frac{\partial F}{\partial y} dy = 0
\end{equation*}
Проведем частное интегрирование по $y$:
\begin{equation*}
F = \int  e^x \left(x^2 + y^2 \right)dy =  e^x \left( x^2 y + \frac{y^3}{3} \right) + C =   e^x \left( x^2 y + \frac{y^3}{3} \right) + C(x)
\end{equation*}
Продифференцируем полученное выражение по $x$:
\begin{equation*}
\frac{\partial F}{\partial x} = \left( e^x \left( x^2 y + \frac{y^3}{3} \right) + C(x) \right)'_x =  e^x \left( x^2 y + \frac{y^3}{3} \right) + 2xy e^x+ C'_x
\end{equation*}
Также:
\begin{equation*}
\frac{\partial F}{\partial x} =  e^x \left( 2xy + x^2y+\frac{y^3}{3} \right)
\end{equation*}
Приравняем 2 выражения выше:

\begin{equation*}
e^x \left( 2xy + x^2y+\frac{y^3}{3} \right)  =  e^x \left( x^2 y + \frac{y^3}{3} \right) + 2xy e^x+ C'_x
\end{equation*}
\begin{equation*}
2xye^x + x^2ye^x+\frac{y^3}{3} e^x  =  x^2 ye^x + \frac{y^3}{3}e^x  + 2xy e^x+ C'_x
\end{equation*}
\begin{equation*}
C'_x = 0
\end{equation*}
\begin{equation*}
C = const
\end{equation*}
При $x = 0, y = 0$ решение входит в общий интеграл.\\\\
\textit{\textbf{Ответ:}} $ e^x \left( x^2 y + \frac{y^3}{3} \right) = C$ -- общий интеграл.

\section{Решить уравнение, подобрав интегрирующий множитель}
\begin{equation*}
\left(2x^3y^2 - y \right)dx + \left(2x^2y^3-x \right)dy = 0
\end{equation*}
Записать общий интеграл уравнения в виде $F(x, y) = C$.\\\\
\textit{\textbf{Решение:}}\\
Попробуем преобразовать выражение так, чтобы получить уравнение в полных дифференциалах:
\begin{equation*}
\left(2x^3y^2 - y \right)dx + \left(2x^2y^3-x \right)dy = 0  \left| \right. : x^2y^2 \neq 0
\end{equation*}
\begin{equation*}
\left(2x - \frac{1}{x^2y} \right)dx + \left(2y-\frac{1}{xy^2} \right)dy = 0 
\end{equation*}
\begin{equation*}
P(x, y) = 2x - \frac{1}{x^2y}
\end{equation*}
\begin{equation*}
Q(x, y) = 2y-\frac{1}{xy^2} 
\end{equation*}
\begin{equation*}
P'_y = \frac{1}{x^2y^2}
\end{equation*}

\begin{equation*}
Q'_x = \frac{1}{x^2y^2} 
\end{equation*}
\begin{equation*}
P'_y= Q'_x 
\end{equation*}
Значит, по признаку полученное уравнение является уравнением в полных дифференциалах и имеет вид:
\begin{equation*}
\frac{\partial F}{\partial x} dx + \frac{\partial F}{\partial y} dy = 0
\end{equation*}
Проведем частное интегрирование по $y$:
\begin{equation*}
F = \int  \left(2y-\frac{1}{xy^2} \right)  dy = y^2 + \frac{1}{xy} + C(x)
\end{equation*}
Продифференцируем полученное выражение по $x$:
\begin{equation*}
\frac{\partial F}{\partial x} = \left(  y^2 + \frac{1}{xy} + C(x)  \right)'_x =  -\frac{1}{x^2y} + C'_x
\end{equation*}
Также:
\begin{equation*}
\frac{\partial F}{\partial x} =  2x - \frac{1}{x^2y}
\end{equation*}
Приравняем 2 выражения выше:

\begin{equation*}
2x - \bcancel{\frac{1}{x^2y}}  =   -\bcancel{\frac{1}{x^2y}} + C'_x
\end{equation*}
\begin{equation*}
 C'_x = 2x
\end{equation*}
\begin{equation*}
\frac{dC}{dx} = 2x
\end{equation*}
\begin{equation*}
\int dC= \int 2x dx
\end{equation*}
\begin{equation*}
C(x) = x^2 + C
\end{equation*}
Получим общий интеграл:
\begin{equation*}
y^2 + \frac{1}{xy} + x^2 =  C
\end{equation*}
\textit{\textbf{Ответ:}} $ y^2 + \frac{1}{xy} + x^2 =  C$ -- общий интеграл.

\section{Решить уравнение, подобрав замену}
\begin{equation*}
y' = \frac{x}{y} \left( 1 + e^{5x^2+3y^2} \right)
\end{equation*}
\end{document}













