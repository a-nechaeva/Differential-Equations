\documentclass[a5paper, 10pt]{article}

% Текст
\usepackage[utf8]{inputenc} % UTF-8 кодировка
\usepackage[russian]{babel} % Русский язык
\usepackage{indentfirst} % красная строка в первом параграфе в главе
% Отображение страниц
\usepackage{geometry} % размеры листа и отступов
\usepackage{listings}
\usepackage{color}

\geometry{
	left=12mm,
	top=25mm,
	right=15mm,
	bottom=17mm,
	marginparsep=0mm,
	marginparwidth=0mm,
	headheight=10mm,
	headsep=7mm,
	nofoot}
\usepackage{afterpage,fancyhdr} % настройка колонтитулов
\pagestyle{fancy}
\fancypagestyle{style}{ % создание нового стиля style
	\fancyhf{} % очистка колонтитулов
	\fancyhead[LO, RE]{Лабораторная работа № 4 } % название документа наверху
	\fancyhead[RO, LE]{Динамические системы} % название section наверху
	\fancyfoot[RO, LE]{\thepage} % номер страницы справа внизу на нечетных и слева внизу на четных
	\renewcommand{\headrulewidth}{0.25pt} % толщина линии сверху
	\renewcommand{\footrulewidth}{0pt} % толцина линии снизу
}
\fancypagestyle{plain}{ % создание нового стиля plain -- полностью пустого
	\fancyhf{}
	\renewcommand{\headrulewidth}{0pt}
}
\fancypagestyle{title}{ % создание нового стиля title -- для титульной страницы
	\fancyhf{}
	\fancyhead[C]{{\footnotesize
			Министерство образования и науки Российской Федерации\\
			Федеральное государственное автономное образовательное учреждение высшего образования
	}}
	\fancyfoot[C]{{\large 
			Санкт-Петербург, 2023-2024
	}}
	\renewcommand{\headrulewidth}{0pt}
}

% Математика
\usepackage{amsmath, amsfonts, amssymb, amsthm} % Набор пакетов для математических текстов
%\usepackage{dmvnbase} % мехматовский пакет latex-сокращений
\usepackage{cancel} % зачеркивание для сокращений
% Рисунки и фигуры
\usepackage[pdftex]{graphicx} % вставка рисунков
\usepackage{wrapfig, subcaption} % вставка фигур, обтекая текст
\usepackage{caption} % для настройки подписей
\captionsetup{figurewithin=none,labelsep=period, font={small,it}} % настройка подписей к рисункам
% Рисование
\usepackage{tikz} % рисование
\usepackage{circuitikz}
\usepackage{pgfplots} % графики
% Таблицы
\usepackage{multirow} % объединение строк
\usepackage{multicol} % объединение столбцов
% Остальное
\usepackage[unicode, pdftex]{hyperref} % гиперссылки
\usepackage{enumitem} % нормальное оформление списков
\setlist{itemsep=0.15cm,topsep=0.15cm,parsep=1pt} % настройки списков
% Теоремы, леммы, определения...
\theoremstyle{definition}
\newtheorem{Def}{Определение}
\newtheorem*{Axiom}{Аксиома}
\theoremstyle{plain}
\newtheorem{Th}{Теорема}
\newtheorem{Lem}{Лемма}
\newtheorem{Cor}{Следствие}
\newtheorem{Ex}{Пример}
\theoremstyle{remark}
\newtheorem*{Note}{Замечание}
\newtheorem*{Solution}{Решение}
\newtheorem*{Proof}{Доказательство}
% Свои команды
\newcommand{\comb}[1]{\left[\hspace{-4pt}\begin{array}{l}#1\end{array}\right.\hspace{-5pt} } % совокупность уравнений
% Титульный лист
\usepackage{csvsimple-l3}
\newcommand*{\titlePage}{
	\thispagestyle{title}
	\begingroup
	\begin{center}
		%		{\footnotesize
			%			Министерство образования и науки Российской Федерации\\
			%			Федеральное государственное автономное образовательное учреждение высшего образования
			%		}
		%		
		\vspace*{6ex}
		
		{\small
			САНКТ-ПЕТЕРБУРГСКИЙ НАЦИОНАЛЬНЫЙ ИССЛЕДОВАТЕЛЬСКИЙ УНИВЕРСИТЕТ ИТМО	
		}
		
		\vspace*{2ex}
		
		{\normalsize
			Факультет систем управления и робототехники
		}
		
		\vspace*{15ex}
		
		{\Large \bfseries 
			Домашняя работа № 4
		}
\vspace*{2ex}
	{\Large \bfseries 
			
"Уравнения высших порядков: понижение порядка, ЛОУ и ЛУ"
		}
\vspace*{2ex}
		
		{\normalsize
			по дисциплине Дифференциальные уравнения
		}

	\end{center}
	\vspace*{20ex}
	\begin{flushright}
		{\large 
			\underline{Выполнила}: студентка гр. \textbf{R3238}\\
			\begin{flushright}
				\textbf{Нечаева А. А.}\\
			\end{flushright}
		}
		
		\vspace*{5ex}
		
		{\large 
			\underline{Преподаватель}: \textit{Магазенков Е. Н.}
		}
	\end{flushright}	
	\newpage
	\setcounter{page}{1}
	\endgroup}

\begin{document}
	\titlePage
	\pagestyle{style}
\newpage
\section{}
\textit{Решить уравнение, понизив порядок}
\\
\begin{equation}
\left( y' + 2y \right) y'' = y'^2
\end{equation}
1. Понижение порядка. Замена: $y'=p, \, \, y''=p'p$:
\begin{equation}
\left( p + 2y \right) p'p = p^2 \, \left|\, \, : p \neq 0 \right.
\end{equation}
Заметим, что $p=0$ -- тоже решение.
\begin{equation}
\left( p + 2y \right) p' = p
\end{equation}
\begin{equation}
\left( p + 2y \right) \frac{dp}{dy} = p
\end{equation}
\begin{equation}
\left( p + 2y \right) dp = pdy
\end{equation}
2. Получено однородное уравнение. Воспользуемся подстановкой: $p = vy \, \to \, dp=vdy+ydv$
\begin{equation}
\left( vy + 2y \right) \left(  vdy+ydv \right) =  vydy
\end{equation}
\begin{equation}
y\left( v + 2 \right) \left(  vdy+ydv \right) =  vydy  \, \left|\, \, : y \neq 0 \right.
\end{equation}
Заметим, что $y=0$ -- тоже решение
\begin{equation}
\left( v + 2 \right) \left(  vdy+ydv \right) =  vdy 
\end{equation}
\begin{equation}
v^2dy+2vdy+vydv+2ydv=vdy
\end{equation}
\begin{equation}
v^2dy+vdy+vydv+2ydv=0
\end{equation}
\begin{equation}
(v^2+v)dy=-y(v+2)dv  \, \left|\, \, : y (v^2+v) \neq 0 \right.
\end{equation}
\begin{equation}
\int \frac{dy}{y} = -\int \frac{v+2}{v^2+v}dv
\end{equation}
\begin{multline}
\int \frac{v+2}{v^2+v}dv = \int \frac{dv}{v+1} + 2\int \frac{dv}{v^2+v} = \int \frac{dv}{v+1} + 2\left(-\int \frac{dv}{v+1}+ \int \frac{dv}{v} \right) = \\
= 2 \ln |v| - \ln |v+1| + C = \ln C \frac{v^2}{v+1}
\end{multline}
\begin{equation}
\ln y = -\ln C \frac{v^2}{v+1}
\end{equation}
\begin{equation}
Cy = \frac{v + 1}{v^2}
\end{equation}
3. Вернемся к переменной $p$: $v = \frac{p}{y}$
\begin{equation}
Cy = \frac{\frac{p}{y} + 1}{\left(\frac{p}{y} \right)^2}
\end{equation}
\begin{equation}
Cy = \frac{\left(\frac{p}{y} + 1 \right) y^2}{p^2}
\end{equation}
\begin{equation}
Cy = \frac{y}{p} + \frac{y^2}{p^2}
\end{equation}
\begin{equation}
Cy = \frac{y}{y'} + \frac{y^2}{(y')^2}
\end{equation}
\begin{equation}
Cy(y')^2 = yy' + y^2
\end{equation}
\begin{equation}
C(y')^2 - y' - y = 0
\end{equation}
Решим, как квадратное уравнение относительно $y'$
\begin{equation}
\sqrt{D} = \sqrt{1 + 4yC}
\end{equation}
\begin{equation}
y'_1 = \frac{-b - \sqrt{D}}{2a} = \frac{1 - \sqrt{1 + 4yC}}{2C}
\end{equation}
\begin{equation}
y'_2 = \frac{-b + \sqrt{D}}{2a} = \frac{1 + \sqrt{1 + 4yC}}{2C}
\end{equation}
В общем виде:
\begin{equation}
y'  = \frac{1 \pm \sqrt{1 + 4yC}}{2C}
\end{equation}
\begin{equation}
\frac{dy}{dx}  = \frac{1 \pm \sqrt{1 + 4yC}}{2C}
\end{equation}
\begin{equation}
  \frac{2C}{1 \pm \sqrt{1 + 4yC}} dy = dx
\end{equation}
\begin{equation}
 \int  \frac{2C}{1 \pm \sqrt{1 + 4yC}} dy = \int dx
\end{equation}
\begin{multline}
 \int  \frac{2C}{1 \pm \sqrt{1 + 4yC}} dy = \left|  \begin{cases} u = 4yC + 1, \\
y = \frac{u-1}{4C},\\
dy = \frac{du}{4C}  \end{cases} \right| = \frac{1}{2} \int \frac{du}{\pm \sqrt{u} + 1} =  \left|  \begin{cases} w = \pm \sqrt{u} + 1, \\
dw = \pm \frac{1}{2}\frac{du}{\sqrt{u}}  \end{cases} \right| =\\=
 \int dw - \int \frac{dw}{w} = \pm \sqrt{4yC + 1} + 1 - \ln \left( \pm \sqrt{4yC + 1} + 1 \right) + A
\end{multline}
\begin{equation}
 \pm \sqrt{4yC + 1} + 1 - \ln \left( \pm \sqrt{4yC + 1} + 1 \right) + A = x
\end{equation}
Объединим $1$ и $A$ и оставим предыдущее обозначение константы $A$:
\begin{equation}
 \pm \sqrt{4yC + 1} - \ln \left( \pm \sqrt{4yC + 1} + 1 \right) + A = x
\end{equation}

\textbf{\textit{Ответ:}} $ \pm \sqrt{4yC + 1} - \ln \left( \pm \sqrt{4yC + 1} + 1 \right) + A = x$

\newpage
\section{}
\textit{Решить ЛОДУ}
\\
\begin{equation}
y^{(4)} - 6y'''+14y''-16y'+8y=0
\end{equation}
Соответствующий характеристический полином:
\begin{equation}
\lambda^4 - 6\lambda^3+14\lambda^2-16\lambda+8=0
\end{equation}
Первый корень $\lambda = 2$ -- кратности 2:
\begin{equation}
\left( \lambda - 2 \right)^2 \left( \lambda^2 - 2\lambda + 2 \right) = 0
\end{equation}
\begin{equation}
\begin{cases}
\lambda_1 = 2\\
\lambda_2 = 2\\
\lambda_3 = 1 - i\\
\lambda_4 = 1 + i
\end{cases}
\end{equation}

\textbf{\textit{Ответ:}} $y = C_1e^{2x} + C_2 x e^{2x} + C_3 e^x \cos x + C_4 e^x \sin x $

\newpage
\section{}
\textit{Решить ЛНДУ методом вариации произвольной постоянной}
\\
\begin{equation}
y''' -y' = \frac{e^x}{1+e^x}
\end{equation}
1. Решение соотвествующего однородного дифференциального уравнения:
\begin{equation}
y''' -y' = 0
\end{equation}
Характеристический полином:
\begin{equation}
\lambda^3 -\lambda = 0
\end{equation}
Его корни:
\begin{equation}
\begin{cases}
\lambda_1 = 0\\
\lambda_2 = -1\\
\lambda_3 = 1
\end{cases}
\end{equation}
Решение:
\begin{equation}
y = C_1 + C_2e^{-x} + C_3 e^{x}
\end{equation}
Решим уравнеие методом вариации произвольной постоянной:
\begin{equation}
y' = C_1' + C_2'e^{-x} -C_2e^{-x} + C_3' e^{x}+ C_3 e^{x}
\end{equation}
При этом $C_1' + C_2'e^{-x}+ C_3' e^{x} = 0$.
\begin{equation}
y'' = -C_2'e^{-x} + C_2e^{-x} + C_3' e^{x}+ C_3 e^{x}
\end{equation}
При этом $- C_2'e^{-x}+ C_3' e^{x} = 0$.
\begin{equation}
y''' = C_2'e^{-x} - C_2e^{-x} + C_3' e^{x}+ C_3 e^{x}
\end{equation}
\begin{multline}
y''' -y' = C_2'e^{-x} - C_2e^{-x} + C_3' e^{x}+ C_3 e^{x} + C_2e^{-x} - C_3 e^{x} =\\=  C_2'e^{-x}+ C_3' e^{x} = \frac{e^x}{1+e^x}
\end{multline}

\begin{equation}
\begin{cases}
C_1' + C_2'e^{-x}+ C_3' e^{x} = 0\\
- C_2'e^{-x}+ C_3' e^{x} = 0\\
C_2'e^{-x}+ C_3' e^{x} = \frac{e^x}{1+e^x}
\end{cases}
\to
\begin{cases}
C_1' + C_2'e^{-x}+ C_3' e^{x} = 0\\
C_3' e^{2x} =  C_2'\\
C_2'e^{-x}+ C_3' e^{x} = \frac{e^x}{1+e^x}
\end{cases}
\end{equation}

\begin{equation}
\begin{cases}
C_1' = - 2C_3' e^{x}\\
C_2' = C_3' e^{2x}\\
 C_3'  = \frac{1}{2}\frac{1}{1+e^x}
\end{cases}
\to
\begin{cases}
C_1' = - \frac{e^x}{1+e^x}\\
C_2' = \frac{1}{2}\frac{ e^{2x}}{1+e^x}\\
C_3'  = \frac{1}{2}\frac{1}{1+e^x}
\end{cases}
\end{equation}

\begin{equation}
C_1' = - \frac{e^x}{1+e^x}  \to C_1 = -ln(e^x+1) + A
\end{equation}
\begin{equation}
C_2' = \frac{1}{2}\frac{ e^{2x}}{1+e^x}  \to C_2 = \frac{e^x - ln (e^x+1)}{2} + B
\end{equation}
\begin{equation}
C_3' = \frac{1}{2}\frac{1}{1+e^x} \to C_3 = \frac{x - ln(e^x+1)}{2} + C
\end{equation}
\textbf{\textit{Ответ:}}\\
$y = -ln(e^x+1) + A + \left(\frac{e^x - ln (e^x+1)}{2} + B  \right)e^{-x} + \left( \frac{x - ln(e^x+1)}{2} + C \right) e^{x}$


\end{document}













