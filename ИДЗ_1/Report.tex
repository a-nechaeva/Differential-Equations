\documentclass[a5paper, 10pt]{article}

% Текст
\usepackage[utf8]{inputenc} % UTF-8 кодировка
\usepackage[russian]{babel} % Русский язык
\usepackage{indentfirst} % красная строка в первом параграфе в главе
% Отображение страниц
\usepackage{geometry} % размеры листа и отступов
\geometry{
	left=12mm,
	top=25mm,
	right=15mm,
	bottom=17mm,
	marginparsep=0mm,
	marginparwidth=0mm,
	headheight=10mm,
	headsep=7mm,
	nofoot}
\usepackage{afterpage,fancyhdr} % настройка колонтитулов
\pagestyle{fancy}
\fancypagestyle{style}{ % создание нового стиля style
	\fancyhf{} % очистка колонтитулов
	\fancyhead[LO, RE]{Индивидуальное домашнее задание №1} % название документа наверху
	%\fancyhead[RO, LE]{\leftmark} % название section наверху
	\fancyfoot[RO, LE]{\thepage} % номер страницы справа внизу на нечетных и слева внизу на четных
	\renewcommand{\headrulewidth}{0.25pt} % толщина линии сверху
	\renewcommand{\footrulewidth}{0pt} % толцина линии снизу
}
\fancypagestyle{plain}{ % создание нового стиля plain -- полностью пустого
	\fancyhf{}
	\renewcommand{\headrulewidth}{0pt}
}
\fancypagestyle{title}{ % создание нового стиля title -- для титульной страницы
	\fancyhf{}
	\fancyhead[C]{{\footnotesize
			Министерство образования и науки Российской Федерации\\
			Федеральное государственное автономное образовательное учреждение высшего образования
	}}
	\fancyfoot[C]{{\large 
			Санкт-Петербург, 2023-2024
	}}
	\renewcommand{\headrulewidth}{0pt}
}

% Математика
\usepackage{epigraph}
\usepackage{cancel}
\usepackage{amsmath, amsfonts, amssymb, amsthm} % Набор пакетов для математических текстов
%\usepackage{dmvnbase} % мехматовский пакет latex-сокращений
\usepackage{cancel} % зачеркивание для сокращений
% Рисунки и фигуры
\usepackage[pdftex]{graphicx} % вставка рисунков
\usepackage{wrapfig, subcaption} % вставка фигур, обтекая текст
\usepackage{caption} % для настройки подписей
\captionsetup{figurewithin=none,labelsep=period, font={small,it}} % настройка подписей к рисункам
% Рисование
\usepackage{tikz} % рисование
\usepackage{circuitikz}
\usepackage{pgfplots} % графики
% Таблицы
\usepackage{multirow} % объединение строк
\usepackage{multicol} % объединение столбцов
% Остальное
\usepackage[unicode, pdftex]{hyperref} % гиперссылки
\usepackage{enumitem} % нормальное оформление списков
\setlist{itemsep=0.15cm,topsep=0.15cm,parsep=1pt} % настройки списков
% Теоремы, леммы, определения...
\theoremstyle{definition}
\newtheorem{Def}{Определение}
\newtheorem*{Axiom}{Аксиома}
\theoremstyle{plain}
\newtheorem{Th}{Теорема}
\newtheorem{Lem}{Лемма}
\newtheorem{Cor}{Следствие}
\newtheorem{Ex}{Пример}
\theoremstyle{remark}
\newtheorem*{Note}{Замечание}
\newtheorem*{Solution}{Решение}
\newtheorem*{Proof}{Доказательство}
% Свои команды
\newcommand{\comb}[1]{\left[\hspace{-4pt}\begin{array}{l}#1\end{array}\right.\hspace{-5pt} } % совокупность уравнений
% Титульный лист
\usepackage{csvsimple-l3}
\newcommand*{\titlePage}{
	\thispagestyle{title}
	\begingroup
	\begin{center}
		%		{\footnotesize
			%			Министерство образования и науки Российской Федерации\\
			%			Федеральное государственное автономное образовательное учреждение высшего образования
			%		}
		%		
		\vspace*{6ex}
		
		{\small
			САНКТ-ПЕТЕРБУРГСКИЙ НАЦИОНАЛЬНЫЙ ИССЛЕДОВАТЕЛЬСКИЙ УНИВЕРСИТЕТ ИТМО	
		}
		
		\vspace*{2ex}
		
		{\normalsize
			Факультет систем управления и робототехники
		}
		
		\vspace*{15ex}
		
		{\Large \bfseries 
			Индивидуальное домашнее задание №1
		}
\vspace*{3ex}
		
		{ \Large 
			Вариант 3
		}
\vspace*{3ex}
		
		{  \bfseries 
			по дисциплине Дифференциальные уравнения
		}
	\end{center}
	\vspace*{20ex}
	\begin{flushright}
		{\large 
			\underline{Выполнила}: студентка гр. \textbf{R3238}\\
			\begin{flushright}
				\textbf{Нечаева А. А.}\\
			\end{flushright}
		}
		
		\vspace*{5ex}
		
		{\large 
			\underline{Преподаватель}: \textit{Бойцев Антон Александрович}
		}
	\end{flushright}	
	\newpage
	\setcounter{page}{1}
	\endgroup}

\begin{document}
	\titlePage
	\pagestyle{style}
	
\newpage
\epigraph{Трудность решения в какой-то мере входит в само понятие задачи: там, где нет трудности, нет и задачи.}{Д. Пойа}

\section{}
Привести заменой $x=z^m$ уравнение 
\begin{equation*}
(xy^2+1)yx' + 2x = 0, \, x > 0, \, y > 0
\end{equation*}
к однородному и решить его. Записать ответ в виде $F(x, y) = C$.\\
\textit{\textbf{Решение:}}\\
Пусть $x=z^m, \, y = z$, тогда $x'=mz'z^{m-1}$, запишем получившееся уравнение:
\begin{equation*}
(z^mz^2+1)zmz'z^{m-1} + 2z^m = 0, \, z^m > 0, \, z > 0
\end{equation*}
\begin{equation*}
mz'z^{2m+2}+mz'z^{m} + 2z^m = 0
\end{equation*}
\begin{equation*}
mz'z^{2m+2}= -(mz' + 2)z^m
\end{equation*}
\begin{equation*}
2m+2 = m \to m = -2
\end{equation*}
Таким образом, $x = z^{-2}, \, x' = -2z' z^{-3}$, подставим в исходное уравнение, получим:
\begin{equation*}
-2z' z^{-3}( z^{-2}y^2+1)y + 2 z^{-2} = 0 \, \left| \right. : -2z^{-2} \neq 0
\end{equation*}
\begin{equation*}
z' z^{-1}( z^{-2}y^2+1)y -1 = 0 
\end{equation*}
\begin{equation*}
z' z^{-3}y^3+ z' z^{-1}y -1 = 0 
\end{equation*}
Полученное уравнение является однородным, убедимся, подставив $z = \lambda z, \, y = \lambda y$:
\begin{equation*}
 F( \lambda z,\, \lambda y ) = z' \lambda^{-3} z^{-3} \lambda^3y^3+ z' \lambda^{-1}z^{-1} \lambda y -1 =  \lambda^0  F(  z,\,  y )
\end{equation*}
Решаем уравнение:
\begin{equation*}
z'  z^{-1}y(z^{-2}y^2+ 1) -1 = 0 
\end{equation*}
Подстановка: $z = ty, \, z' = t'y + t$
\begin{equation*}
(t'y + t)  (ty)^{-1}y((ty)^{-2}y^2+ 1) -1 = 0 
\end{equation*}
\begin{equation*}
(t'y + t) t^{-1}(t^{-2}+ 1) -1 = 0 
\end{equation*}
\begin{equation*}
t'y + t= \frac{t}{t^{-2}+ 1}, \, \, t^{-2}+ 1 \neq 0 \, (\textit{из условия всегда выполнено})
\end{equation*}
\begin{equation*}
t'y = \frac{t}{t^{-2}+ 1} - t
\end{equation*}
\begin{equation*}
\frac{dt}{dy}y = \frac{t}{t^{-2}+ 1} - t
\end{equation*}
\begin{equation*}
\frac{dt}{dy}y = \frac{t - t^{-1}- t }{t^{-2}+ 1} 
\end{equation*}
\begin{equation*}
\frac{dt}{dy}y = \frac{- t^{-1}}{t^{-2}+ 1} 
\end{equation*}
\begin{equation*}
-t(t^{-2}+ 1) \, dt  = \frac{dy}{y}
\end{equation*}
\begin{equation*}
-(t^{-1} + t) \, dt  = \frac{dy}{y}
\end{equation*}
\begin{equation*}
- \ln t - \frac{t^2}{2}  = \ln y + C
\end{equation*}
\begin{equation*}
- \ln \frac{z}{y} - \frac{z^2}{2y^2}  = \ln y + C
\end{equation*}
\begin{equation*}
- \ln z +  \ln y - \frac{z^2}{2y^2}  = \ln y + C
\end{equation*}

\begin{equation*}
- \ln z  - \frac{z^2}{2y^2}  =  C
\end{equation*}
Обратная замена: $x = z^{-2} \to z = \frac{1}{\sqrt{x}}$
\begin{equation*}
- \ln \frac{1}{\sqrt{x}}   - \frac{1}{2xy^2}  =  C
\end{equation*}
\begin{equation*}
 \ln \sqrt{x}   - \frac{1}{2xy^2}  =  C
\end{equation*}

\textit{\textbf{Ответ:}} $\ln \sqrt{x}   - \frac{1}{2xy^2}  =  C$



\newpage
\section{}
Решить линейное уравнение методом вариации произвольных постоянных (методом Лагранжа). Пользуясь формулой общего решения линейного уравнения, проверьте полученный ответ.\\
Записать ответ в виде $y = f(x, C).$
\begin{equation*}
y' = \frac{2y}{x \ln x} + \frac{1}{x}, \, x > 1\\
\end{equation*}
\textit{\textbf{Решение:}}\\


\textit{\textbf{Ответ:}}

\newpage
\section{}
Привести уравнение Риккати к линейному. решить полученное линейное уравнение, используя метод интегрирующего множителя.\\
Записать ответ в виде $F(x, y) = C$.
\begin{equation*}
xy' = x^3 + (1 - 2x^2)y + xy^2
\end{equation*}
\textit{\textbf{Решение:}}\\


\textit{\textbf{Ответ:}}


\newpage
\section{}
Решить уравнение в дифференциалах, подобрав интегрирующий множитель в виде $\mu (x, y) = (x+y^2)^{\alpha}$.\\
Записать ответ в виде $F(x, y) = C$.
\begin{equation*}
2y(x + y^2 - 1) dy + (x^2y^2+x^3 - 1)dx = 0
\end{equation*}
\textit{\textbf{Решение:}}\\


\textit{\textbf{Ответ:}}

\newpage
\section{}
Решить уравнение методом введения параметра. \\
Записать ответ в виде $x = f(y, C)$.\\
Исследовать на наличие особых решений. Построить на одной координатной плоскости графики нескольких интегральных кривых и, при наличии, особых решений.
\begin{equation*}
2x = \frac{y}{y'} + \ln (yy'), \, y >0\\
\end{equation*}
\textit{\textbf{Решение:}}\\


\textit{\textbf{Ответ:}}


\begin{equation*}
\frac{\partial F}{\partial x} =  2x - \frac{1}{x^2y}
\end{equation*}



\end{document}













